
% Carátula para el manual del operador
\documentclass[a4paper,12pt]{article}
\usepackage[spanish]{babel}
\usepackage[utf8]{inputenc}
\usepackage{graphicx}
\usepackage{geometry}
\usepackage{xcolor}
\usepackage[colorlinks=true, linkcolor=black, urlcolor=black]{hyperref}
\geometry{top=2.5cm, bottom=2.5cm, left=2.5cm, right=2.5cm}

\begin{document}

% Logo arriba a la derecha
\begin{flushright}
    \includegraphics[width=4cm]{LOGO-EEST1.png}
\end{flushright}

\vspace*{2cm}

\begin{center}
    {\LARGE\bfseries Manual del Operador} \\[1.5cm]
    \textbf{Integrantes:} \\[0.3cm]
    Santiago Ibarra \\[0.2cm]
    Agustin Colman \\[0.2cm]
    Carlos Insaurralde \\[0.2cm]
    Gabriel Beneitez \\[1.2cm]
    Curso: 7°2 \\
    \vspace{0.5cm}
    \textbf{Materia:} Proyecto de Desarrollo de Software para Plataformas Móviles
\end{center}

\vfill

\begin{center}
    \textit{Escuela de Educación Secundaria Técnica N°1} \\
    \textit{Año 2025}
\end{center}

\thispagestyle{empty}
\newpage

% Índice de contenidos con enlaces
\tableofcontents
\newpage

% ===============================
% 1. Introducción
% ===============================
\section{Introducción}\label{sec:introduccion}
Este manual está dirigido a operadores y usuarios que deseen poner en funcionamiento el sistema de Reproductor Inteligente de música. Aquí se detallan los pasos para la correcta instalación y puesta en marcha del sistema, así como la ubicación de los archivos necesarios.

% ===============================
% 2. Requisitos y Preparación
% ===============================
\section{Requisitos y Preparación}\label{sec:requisitos}
\begin{itemize}
    \item \textbf{Sistema operativo:} Windows
    \item \textbf{XAMPP:} Instalado y configurado (incluyendo Apache y MySQL)
    \item \textbf{Google Drive:} Acceso para descargar la carpeta de canciones
    \item \textbf{GitHub:} Acceso para descargar los archivos del proyecto
\end{itemize}

% ===============================
% 3. Estructura de Carpetas y Archivos
% ===============================
\section{Estructura de Carpetas y Archivos}\label{sec:estructura}
El proyecto debe ubicarse en la siguiente ruta:

\begin{itemize}
    \item \textbf{Ruta principal:} \\C:\textbackslash xampp\textbackslash htdocs\textbackslash proyectomoviles
\end{itemize}

\begin{verbatim}
C:\xampp\htdocs\proyectomoviles
|-- ppm_music.html / ppm_music.js
|-- list_songs.php
|-- get_songs.php
|-- songs_database.sql
|-- ...
|-- (otras carpetas y archivos del proyecto)
\end{verbatim}

% ===============================
% 4. Descarga de Archivos Necesarios
% ===============================
\section{Descarga de Archivos Necesarios}\label{sec:descarga}
\subsection{Archivos del Proyecto}\label{subsec:archivos}
Los archivos principales del sistema deben descargarse desde el repositorio de GitHub:

\begin{itemize}
    \item \textbf{Repositorio:} \\ \url{https://github.com/SantiagoIbarraa/ProyectoAPIMoviles}
    \item Descargar y copiar todo el contenido en la carpeta mencionada anteriormente.
\end{itemize}

\subsection{Carpeta de Canciones}\label{subsec:canciones}
La carpeta con las canciones debe descargarse desde Google Drive:

\begin{itemize}
    \item \textbf{Enlace:} \\ \url{https://drive.google.com/file/d/1PMr6eChcD50MDNXACt3o5pAC62_7IB9x/view?usp=sharing}
    \item Extraer la carpeta de canciones y colocarla dentro de la ruta principal del proyecto.
\end{itemize}

% ===============================
% 5. Set-Up y Puesta en Marcha
% ===============================
\section{Set-Up y Puesta en Marcha}\label{sec:setup}
\begin{enumerate}
    \item Instalar XAMPP y asegurarse de que los servicios de Apache y MySQL estén activos.
    \item Descargar los archivos del proyecto desde GitHub y copiarlos en \\C:\textbackslash xampp\textbackslash htdocs\textbackslash proyectomoviles.
    \item Descargar la carpeta de canciones desde Google Drive y colocarla en la misma carpeta del proyecto.
    \item Importar el archivo \texttt{songs\_database.sql} en MySQL usando phpMyAdmin (incluido en XAMPP):
        \begin{itemize}
            \item Acceder a \url{http://localhost/phpmyadmin}
            \item Crear una nueva base de datos (por ejemplo, \texttt{songs})
            \item Importar el archivo \texttt{songs\_database.sql}
        \end{itemize}
    \item Abrir el navegador y acceder a \url{http://localhost/proyectomoviles/ppm_music.html} para usar el sistema.
\end{enumerate}

% ===============================
% 6. Recomendaciones
% ===============================
\section{Recomendaciones}\label{sec:recomendaciones}
\begin{itemize}
    \item No modificar la estructura de carpetas ni los nombres de archivos.
    \item Realizar copias de seguridad periódicas de la base de datos y la carpeta de canciones.
    \item Ante cualquier inconveniente, revisar la configuración de XAMPP y los permisos de carpetas.
\end{itemize}

\end{document}