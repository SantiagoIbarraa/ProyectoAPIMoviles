% Manual del Administrador
\documentclass[a4paper,12pt]{article}
\usepackage[utf8]{inputenc}
\usepackage[T1]{fontenc}
\usepackage{lmodern}
\usepackage[spanish,es-noshorthands]{babel}
\usepackage{graphicx}
\usepackage{fancyhdr}
\usepackage{hyperref}
\usepackage{geometry}
\usepackage{xcolor}
\usepackage{listings}
\geometry{a4paper, margin=2.5cm}

% Configuración de encabezados y pies de página
\pagestyle{fancy}
\fancyhf{}
\renewcommand{\headrulewidth}{0pt}
\renewcommand{\footrulewidth}{0pt}
\fancyhead[R]{\includegraphics[height=1.5cm]{apipmusiclogo.png}}
\fancyfoot[C]{\thepage}
\fancyfoot[R]{Reproductor Inteligente}

% Para la primera página (portada)
\fancypagestyle{firstpage}{
  \fancyhf{}
  \renewcommand{\headrulewidth}{0pt}
  \renewcommand{\footrulewidth}{0.4pt}
  \fancyfoot[C]{\thepage}
  \fancyfoot[R]{Reproductor Inteligente}
}

% Configuración para bloques de código
\lstset{
  basicstyle=\small\ttfamily,
  breaklines=true,
  frame=single,
  numbers=left,
  numberstyle=\tiny,
  keywordstyle=\color{blue},
  commentstyle=\color{green!40!black},
  stringstyle=\color{red},
  showstringspaces=false
}

\begin{document}

% Logo arriba a la derecha
\begin{flushright}
    \includegraphics[width=4cm]{LOGO-EEST1.png}
\end{flushright}

\vspace*{2cm}

\begin{center}
    {\LARGE\bfseries Manual del Administrador} \\[1.5cm]
    \textbf{Integrantes:} \\[0.3cm]
    Santiago Ibarra \\[0.2cm]
    Agustin Colman \\[0.2cm]
    Carlos Insaurralde \\[0.2cm]
    Gabriel Beneitez \\[1.2cm]
    Curso: 7\textdegree2 \\
    \vspace{0.5cm}
    \textbf{Materia:} Proyecto de Desarrollo de Software para Plataformas Móviles
    
    \vspace{0.7cm}
    \includegraphics[width=0.3\textwidth]{apipmusiclogo.png}\\[0.5cm]
\end{center}

\vfill

\begin{center}
    \textit{Escuela de Educación Secundaria Técnica N\textdegree1} \\
    \textit{Año 2025}
\end{center}

\thispagestyle{empty}
\newpage

% Índice de contenidos con enlaces
\tableofcontents
\newpage

\section{Introducción}
\subsection{Propósito del Manual}
Este manual está dirigido a los administradores del sistema Reproductor Inteligente. Proporciona instrucciones detalladas sobre cómo acceder al panel de administración, gestionar la base de datos de canciones, añadir nuevas canciones y realizar otras tareas administrativas.

\subsection{Funciones del Administrador}
Como administrador del Reproductor Inteligente, tendrás acceso a las siguientes funcionalidades:
\begin{itemize}
  \item Acceso al panel de administración mediante credenciales seguras
  \item Gestión completa del catálogo de canciones (añadir, editar, eliminar)
  \item Análisis de archivos de audio para extraer características musicales
  \item Monitoreo del sistema y gestión de usuarios
  \item Configuración de parámetros del sistema
\end{itemize}

\section{Acceso al Panel de Administración}
\subsection{Inicio de Sesión}
Para acceder al panel de administración, sigue estos pasos:
\begin{enumerate}
  \item Abre tu navegador web y navega a: \texttt{http://localhost/proyectomoviles/admin.php}
  \item Se mostrará una pantalla de inicio de sesión como la siguiente:
  \begin{center}
    \fbox{\parbox{0.8\textwidth}{\centering\textbf{Inicio de Sesión - Reproductor Inteligente}\\[0.5cm]
    Usuario: [campo de texto]\\[0.3cm]
    Contraseña: [campo de texto]\\[0.5cm]
    [Botón de Iniciar Sesión]}}
  \end{center}
  \item Ingresa tus credenciales de administrador:
  \begin{itemize}
    \item Usuario: admin
    \item Contraseña: la contraseña proporcionada por el administrador del sistema
  \end{itemize}
  \item Haz clic en "Iniciar Sesión"
\end{enumerate}

\subsection{Consideraciones de Seguridad}
\begin{itemize}
  \item Cambia la contraseña predeterminada después del primer inicio de sesión
  \item No compartas tus credenciales con personas no autorizadas
  \item Cierra sesión cuando no estés utilizando el panel de administración
  \item Utiliza una contraseña segura que combine letras, números y símbolos
\end{itemize}

\section{Panel de Administración}
\subsection{Interfaz Principal}
Una vez que hayas iniciado sesión, serás redirigido al panel de administración principal (\texttt{admin.php}). La interfaz está organizada de la siguiente manera:
\begin{itemize}
  \item \textbf{Barra de navegación superior:} Contiene el logo del sistema, acceso a diferentes secciones y opción de cerrar sesión
  \item \textbf{Panel lateral:} Menú con acceso a todas las funcionalidades administrativas
  \item \textbf{Área principal:} Muestra el contenido de la sección seleccionada
  \item \textbf{Tabla de canciones:} Lista todas las canciones en la base de datos con opciones para editar o eliminar
\end{itemize}

\subsection{Secciones Principales}
\begin{itemize}
  \item \textbf{Dashboard:} Muestra estadísticas generales del sistema
  \item \textbf{Gestión de Canciones:} Permite administrar el catálogo musical
  \item \textbf{Análisis de Audio:} Herramientas para analizar archivos de audio
  \item \textbf{Configuración:} Ajustes generales del sistema
  \item \textbf{Usuarios:} Gestión de cuentas de usuario (si aplica)
\end{itemize}

\section{Gestión de Canciones}
\subsection{Visualización del Catálogo}
El panel principal muestra una tabla con todas las canciones en la base de datos. Para cada canción, se muestra la siguiente información:
\begin{itemize}
  \item ID de la canción
  \item Título
  \item Artista
  \item Año
  \item PPM (Pulsaciones Por Minuto)
  \item Energía
  \item Dinámica
  \item Brillo
  \item Complejidad
  \item Ritmo
  \item Acciones disponibles (reproducir, editar, eliminar)
\end{itemize}

\subsection{Añadir Nuevas Canciones}
Para añadir una nueva canción al sistema:
\begin{enumerate}
  \item En el panel de administración, haz clic en el botón "Añadir Nueva Canción"
  \item Se abrirá un formulario con los siguientes campos:
  \begin{itemize}
    \item \textbf{Archivo de Audio:} Selecciona el archivo MP3 o WAV a subir
    \item \textbf{Título:} Ingresa el título de la canción
    \item \textbf{Artista:} Ingresa el nombre del artista o banda
    \item \textbf{Año:} Ingresa el año de lanzamiento
    \item \textbf{Género:} Selecciona o ingresa el género musical
  \end{itemize}
  \item Haz clic en "Analizar" para procesar el archivo de audio
  \item El sistema analizará automáticamente el archivo y extraerá las siguientes características:
  \begin{itemize}
    \item PPM/BPM (tempo)
    \item Dinámica (variación de volumen)
    \item Brillo (presencia de frecuencias altas)
    \item Complejidad (variaciones en la estructura musical)
    \item Ritmo (combinación de BPM y energía)
  \end{itemize}
  \item Revisa los resultados del análisis y ajusta los valores si es necesario
  \item Haz clic en "Guardar" para añadir la canción a la base de datos
\end{enumerate}

\subsection{Proceso de Análisis de Audio}
El sistema utiliza tecnología avanzada para analizar los archivos de audio:
\begin{enumerate}
  \item El archivo se carga en el servidor
  \item Se procesa utilizando la Web Audio API
  \item Se extraen características como BPM, energía, dinámica, brillo, complejidad y ritmo
  \item Los resultados se muestran en la interfaz para su revisión
  \item Al guardar, los datos se almacenan en la base de datos junto con la ruta al archivo de audio
\end{enumerate}

\subsection{Editar Canciones Existentes}
Para editar una canción ya existente en el sistema:
\begin{enumerate}
  \item En la tabla de canciones, localiza la canción que deseas editar
  \item Haz clic en el botón "Editar" (icono de lápiz) junto a la canción
  \item Se abrirá un formulario similar al de añadir canción, pero con los datos actuales
  \item Modifica los campos necesarios
  \item Si deseas reanalizar el audio, puedes cargar un nuevo archivo
  \item Haz clic en "Guardar Cambios" para actualizar la información
\end{enumerate}

\subsection{Eliminar Canciones}
Para eliminar una canción del sistema:
\begin{enumerate}
  \item En la tabla de canciones, localiza la canción que deseas eliminar
  \item Haz clic en el botón "Eliminar" (icono de papelera) junto a la canción
  \item Confirma la acción en el cuadro de diálogo que aparece
  \item La canción será eliminada de la base de datos y el archivo de audio se eliminará del servidor
\end{enumerate}

\section{Características Avanzadas}
\subsection{Análisis Manual de Audio}
El sistema permite realizar análisis manuales de archivos de audio sin necesidad de añadirlos inmediatamente a la base de datos:
\begin{enumerate}
  \item En el panel lateral, selecciona "Análisis de Audio"
  \item Carga un archivo de audio utilizando el botón "Seleccionar Archivo"
  \item Haz clic en "Analizar"
  \item El sistema mostrará los resultados del análisis con visualizaciones
  \item Puedes guardar estos resultados o utilizarlos como referencia
\end{enumerate}

\subsection{Importación y Exportación de Datos}
Para gestionar grandes cantidades de datos:
\begin{itemize}
  \item \textbf{Exportar catálogo:} Permite descargar un archivo CSV o JSON con toda la información del catálogo musical
  \item \textbf{Importar canciones:} Permite cargar múltiples canciones a través de un archivo CSV con metadatos
\end{itemize}

\section{Mantenimiento del Sistema}
\subsection{Gestión de Archivos de Audio}
\begin{itemize}
  \item Los archivos de audio se almacenan en la carpeta \texttt{/uploads/audio/}
  \item Asegúrate de que esta carpeta tenga permisos de escritura adecuados
  \item Realiza copias de seguridad periódicas de esta carpeta
  \item Monitorea el espacio en disco disponible para evitar problemas de almacenamiento
\end{itemize}

\subsection{Optimización de la Base de Datos}
\begin{itemize}
  \item Realiza mantenimiento regular de la base de datos
  \item Elimina registros huérfanos (canciones sin archivo de audio asociado)
  \item Optimiza las tablas para mejorar el rendimiento
  \item Realiza copias de seguridad periódicas de la base de datos
\end{itemize}

\section{Solución de Problemas}
\subsection{Problemas Comunes y Soluciones}
\begin{itemize}
  \item \textbf{Error al iniciar sesión:} Verifica las credenciales y asegúrate de que la base de datos esté funcionando correctamente
  \item \textbf{Error al subir archivos:} Comprueba los permisos de la carpeta de uploads y el tamaño máximo permitido
  \item \textbf{Análisis de audio fallido:} Verifica que el formato del archivo sea compatible (MP3, WAV) y no esté corrupto
  \item \textbf{Lentitud en el panel de administración:} Optimiza la base de datos y considera eliminar canciones antiguas o no utilizadas
\end{itemize}

\subsection{Registro de Errores}
El sistema mantiene registros de errores que pueden ayudar a diagnosticar problemas:
\begin{itemize}
  \item Los logs de PHP se encuentran en \texttt{/logs/php\_errors.log}
  \item Los logs del servidor web se encuentran en \texttt{/xampp/apache/logs/}
  \item Consulta estos archivos cuando encuentres problemas para obtener más información
\end{itemize}

\section{Apéndice}
\subsection{Estructura de la Base de Datos}
La tabla principal \texttt{songs} tiene la siguiente estructura:
\begin{lstlisting}[language=SQL]
CREATE TABLE songs (
  id INT AUTO_INCREMENT PRIMARY KEY,
  name VARCHAR(255) NOT NULL,
  artist VARCHAR(255) NOT NULL,
  year INT,
  genre VARCHAR(100),
  PPM INT,
  energy FLOAT,
  dynamics FLOAT,
  brightness FLOAT,
  complexity FLOAT,
  rhythm FLOAT,
  danceability FLOAT,
  happiness FLOAT,
  audioUrl VARCHAR(255) NOT NULL,
  created_at TIMESTAMP DEFAULT CURRENT_TIMESTAMP,
  updated_at TIMESTAMP DEFAULT CURRENT_TIMESTAMP ON UPDATE CURRENT_TIMESTAMP
);
\end{lstlisting}

\subsection{Archivos Principales del Sistema}
\begin{itemize}
  \item \texttt{admin.php:} Panel de administración principal
  \item \texttt{admin\_add\_song.php:} Procesamiento de nuevas canciones
  \item \texttt{admin\_edit\_song.php:} Edición de canciones existentes
  \item \texttt{admin\_delete\_song.php:} Eliminación de canciones
  \item \texttt{analyze.html:} Interfaz para análisis de audio
  \item \texttt{js/simple-audio-analyzer.js:} Biblioteca para análisis de audio
  \item \texttt{save\_analysis.php:} Guarda los resultados del análisis en la base de datos
\end{itemize}

\end{document}
