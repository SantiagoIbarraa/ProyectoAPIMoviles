\documentclass[a4paper,12pt]{article}
\usepackage[utf8]{inputenc}
\usepackage[T1]{fontenc}
\usepackage{lmodern}
\usepackage[spanish,es-noshorthands]{babel}
\usepackage{graphicx}
\usepackage{fancyhdr}
\usepackage{hyperref}
\usepackage{geometry}
\usepackage{listings}
\usepackage{color}
\usepackage{xcolor}
\geometry{a4paper, margin=2.5cm}

% Configuración de encabezados y pies de página
\pagestyle{fancy}
\fancyhf{}
\renewcommand{\headrulewidth}{0pt}
\renewcommand{\footrulewidth}{0pt}
\fancyhead[R]{\includegraphics[height=1.5cm]{apipmusiclogo.png}}
\fancyfoot[C]{\thepage}
\fancyfoot[R]{Reproductor Inteligente}

% Configuración para bloques de código
\lstset{
  basicstyle=\small\ttfamily,
  breaklines=true,
  frame=single,
  numbers=left,
  numberstyle=\tiny,
  keywordstyle=\color{blue},
  commentstyle=\color{green!40!black},
  stringstyle=\color{red},
  showstringspaces=false
}

\begin{document}

% Logo arriba a la derecha
\begin{flushright}
    \includegraphics[width=4cm]{LOGO-EEST1.png}
\end{flushright}

\vspace*{2cm}

\begin{center}
    {\LARGE\bfseries Manual del Sistema de Administración} \\[1.5cm]
    \textbf{Integrantes:} \\[0.3cm]
    Santiago Ibarra \\[0.2cm]
    Agustin Colman \\[0.2cm]
    Carlos Insaurralde \\[0.2cm]
    Gabriel Beneitez \\[1.2cm]
    Curso: 7\textdegree2 \\
    \vspace{0.5cm}
    \textbf{Materia:} Proyecto de Desarrollo de Software para Plataformas Móviles
    
    \vspace{0.7cm}
    \includegraphics[width=0.3\textwidth]{apipmusiclogo.png}\\[0.5cm]
\end{center}

\vfill

\begin{center}
    \textit{Escuela de Educación Secundaria Técnica N\textdegree1} \\
    \textit{Año 2025}
\end{center}

\thispagestyle{empty}
\newpage

% Índice de contenidos con enlaces
\tableofcontents
\newpage

\section{Sistema de Administración}
\subsection{Descripción General}
El sistema de administración permite gestionar la biblioteca de canciones mediante un panel protegido por autenticación. Solo los usuarios con credenciales de administrador pueden acceder a estas funcionalidades.

\subsection{Archivos Relacionados}
\begin{itemize}
    \item \texttt{login.php} - Maneja la autenticación de administradores
    \item \texttt{admin.php} - Panel de administración principal
    \item \texttt{admin\_add\_song.php} - Procesa la adición de nuevas canciones
    \item \texttt{admin\_delete\_song.php} - Procesa la eliminación de canciones
    \item \texttt{admin\_get\_songs.php} - Obtiene la lista de canciones para el panel de administración
\end{itemize}

\subsection{Detalles de Implementación}
\subsubsection{admin.php}
Este archivo constituye la interfaz gráfica del panel de administración, permitiendo agregar y gestionar canciones en el Reproductor Inteligente.

\paragraph{Características principales:}
\begin{itemize}
    \item Control de acceso mediante sesiones PHP
    \item Formulario para agregar nuevas canciones con todos sus parámetros
    \item Listado de canciones existentes con opciones para administrarlas
    \item Interfaz para eliminar canciones
\end{itemize}

\paragraph{Estructura del código:}
El archivo se estructura en las siguientes secciones principales:

\begin{enumerate}
    \item \textbf{Verificación de sesión:} Al inicio del archivo se verifica si el usuario está autenticado como administrador mediante \texttt{\$\_SESSION['admin\_logged\_in']}. Si no está autenticado, se redirige a la página principal.
    
    \begin{lstlisting}[language=PHP]
<?php
session_start();

// Verificar si el usuario esta autenticado como administrador
if (!isset($_SESSION['admin_logged_in']) || $_SESSION['admin_logged_in'] !== true) {
    // Si no esta autenticado, redirigir a la pagina principal
    header('Location: index.html');
    exit;
}
?>
    \end{lstlisting}
    
    \item \textbf{Interfaz de usuario:} Implementa un panel completo con:
    \begin{itemize}
        \item Header con navegación y botón de cierre de sesión
        \item Formulario para agregar canciones con campos para:
        \begin{itemize}
            \item Datos básicos (título, artista, año)
            \item Parámetros tradicionales (energía, bailabilidad, felicidad, instrumentalidad)
            \item Características avanzadas (dinámica, brillo, complejidad, ritmo)
            \item Carga de archivos de audio
        \end{itemize}
    \end{itemize}
    
    \item \textbf{Tabla de administración:} Muestra todas las canciones con opciones para:
    \begin{itemize}
        \item Buscar por título o artista
        \item Ordenar por diferentes campos
        \item Eliminar canciones
    \end{itemize}
    
    \item \textbf{JavaScript para interactividad:} Incluye funciones para:
    \begin{itemize}
        \item Actualizar valores de los sliders en tiempo real
        \item Enviar formularios mediante AJAX
        \item Manejar la eliminación de canciones con confirmación
        \item Filtrar y ordenar la lista de canciones
    \end{itemize}
\end{enumerate}

\subsubsection{admin\_add\_song.php}
Este archivo procesa la solicitud de agregar una nueva canción, valida los datos, guarda el archivo de audio y registra la información en la base de datos.

\paragraph{Características principales:}
\begin{itemize}
    \item Validación completa de datos y archivos
    \item Manejo de errores con respuestas JSON
    \item Soporte para parámetros tradicionales y avanzados
    \item Almacenamiento de archivos de audio
\end{itemize}

\paragraph{Estructura del código:}
\begin{enumerate}
    \item \textbf{Configuración inicial:}
    \begin{itemize}
        \item Establece cabeceras para respuesta JSON
        \item Define variables de conexión a la base de datos
        \item Configura el directorio para almacenar archivos de audio
    \end{itemize}
    
    \begin{lstlisting}[language=PHP]
<?php
header('Content-Type: application/json');

$host = 'localhost';
$dbname = 'songs_database';
$user = 'root';
$pass = '';

// Directorio donde se guardaran los archivos de audio
$uploadDir = 'SpotiDownloader.com - Las 100 mejores canciones de la historia del Rock/';
    \end{lstlisting}
    
    \item \textbf{Validación de datos:}
    \begin{itemize}
        \item Verifica que se haya enviado un archivo de audio
        \item Comprueba que se hayan proporcionado título y artista
        \item Valida el formato del archivo (mp3, wav, ogg)
        \item Verifica el tamaño máximo (20MB)
    \end{itemize}
    
    \item \textbf{Procesamiento del archivo:}
    \begin{itemize}
        \item Genera un nombre para el archivo basado en el título
        \item Mueve el archivo al directorio de destino
    \end{itemize}
    
    \item \textbf{Procesamiento de parámetros:}
    \begin{itemize}
        \item Extrae los parámetros tradicionales (energía, bailabilidad, etc.)
        \item Extrae las características avanzadas (dinámica, brillo, etc.)
        \item Implementa lógica para usar características avanzadas como sustitutos de parámetros tradicionales si es necesario
    \end{itemize}
    
    \begin{lstlisting}[language=PHP]
// Si no se proporcionan los parametros tradicionales, usar los nuevos como sustitutos
if ($energy === null && $rhythm !== null) {
    $energy = $rhythm; // Usar ritmo como sustituto de energia
}
if ($danceability === null && $dynamics !== null) {
    $danceability = $dynamics; // Usar dinamica como sustituto de bailabilidad
}
if ($happiness === null && $brightness !== null) {
    $happiness = $brightness; // Usar brillo como sustituto de felicidad
}
if ($instrumentalness === null && $complexity !== null) {
    $instrumentalness = 1 - $complexity; // Usar inverso de complejidad como instrumentalidad
}
    \end{lstlisting}
    
    \item \textbf{Inserción en la base de datos:}
    \begin{itemize}
        \item Prepara y ejecuta la consulta SQL para insertar la canción
        \item Devuelve una respuesta JSON con el resultado de la operación
    \end{itemize}
    
    \item \textbf{Manejo de errores:}
    \begin{itemize}
        \item Captura excepciones durante el proceso
        \item Elimina el archivo si se produjo un error
        \item Devuelve mensajes de error detallados
    \end{itemize}
\end{enumerate}

\subsubsection{admin\_delete\_song.php}
Este archivo elimina una canción del sistema, tanto de la base de datos como el archivo físico.

\paragraph{Características principales:}
\begin{itemize}
    \item Recibe datos en formato JSON
    \item Elimina el archivo físico de audio
    \item Elimina el registro de la base de datos
    \item Proporciona respuestas JSON con el resultado
\end{itemize}

\paragraph{Estructura del código:}
\begin{enumerate}
    \item \textbf{Recepción de datos:}
    \begin{itemize}
        \item Decodifica los datos JSON recibidos
        \item Valida que se haya proporcionado un ID de canción
    \end{itemize}
    
    \begin{lstlisting}[language=PHP]
<?php
header('Content-Type: application/json');

// Obtener datos de la solicitud
$data = json_decode(file_get_contents('php://input'), true);

if (!isset($data['id']) || empty($data['id'])) {
    http_response_code(400);
    echo json_encode([
        'success' => false,
        'message' => 'ID de la cancion no proporcionado'
    ]);
    exit;
}
    \end{lstlisting}
    
    \item \textbf{Proceso de eliminación:}
    \begin{itemize}
        \item Primero obtiene la información de la canción, especialmente la URL del audio
        \item Verifica si el archivo existe y lo elimina del sistema de archivos
        \item Elimina el registro de la base de datos mediante una consulta SQL
    \end{itemize}
    
    \item \textbf{Manejo de respuestas:}
    \begin{itemize}
        \item Devuelve una respuesta JSON con el resultado de la operación
        \item Incluye manejo de errores para casos como canción no encontrada
    \end{itemize}
\end{enumerate}

\subsubsection{admin\_get\_songs.php}
Este archivo recupera todas las canciones de la base de datos para mostrarlas en el panel de administración.

\paragraph{Características principales:}
\begin{itemize}
    \item Consulta a la base de datos para obtener todas las canciones
    \item Procesa las rutas de audio para asegurar que sean accesibles
    \item Devuelve los datos en formato JSON
\end{itemize}

\paragraph{Estructura del código:}
\begin{enumerate}
    \item \textbf{Consulta a la base de datos:}
    \begin{itemize}
        \item Establece conexión con la base de datos
        \item Ejecuta una consulta SQL para obtener todas las canciones con sus IDs, artistas, nombres, años, PPM y URLs de audio
        \item Ordena los resultados por nombre
    \end{itemize}
    
    \begin{lstlisting}[language=PHP]
// Obtener todas las canciones con sus IDs para administracion
$stmt = $pdo->query("SELECT id, artist, name, year, PPM as ppm, audioUrl FROM songs ORDER BY name ASC");
$songs = $stmt->fetchAll(PDO::FETCH_ASSOC);
    \end{lstlisting}
    
    \item \textbf{Procesamiento de rutas:}
    \begin{itemize}
        \item Verifica que las rutas de audio estén en formato correcto para ser accesibles desde el navegador
        \item Mantiene las rutas relativas para compatibilidad
    \end{itemize}
    
    \item \textbf{Respuesta:}
    \begin{itemize}
        \item Devuelve los datos en formato JSON para ser procesados por el frontend
    \end{itemize}
\end{enumerate}

\subsection{Flujo de Trabajo del Sistema de Administración}
El sistema de administración sigue el siguiente flujo de trabajo:

\begin{enumerate}
    \item El usuario accede a la página principal (index.html) y hace clic en el botón "Administrador" en el encabezado.
    \item Se muestra un modal de inicio de sesión donde debe ingresar credenciales específicas.
    \item Las credenciales se envían a login.php, que verifica su validez.
    \item Si las credenciales son correctas, se establece una sesión de administrador y se redirige a admin.php.
    \item En admin.php, el administrador puede:
    \begin{itemize}
        \item Agregar nuevas canciones mediante el formulario, que envía los datos a admin\_add\_song.php
        \item Ver la lista de canciones existentes, cargadas mediante admin\_get\_songs.php
        \item Eliminar canciones, lo que envía una solicitud a admin\_delete\_song.php
    \end{itemize}
    \item Al finalizar, el administrador puede cerrar sesión haciendo clic en el botón correspondiente.
\end{enumerate}

\subsection{Consideraciones de Seguridad}
\begin{itemize}
    \item El sistema utiliza sesiones PHP para controlar el acceso al panel de administración.
    \item Las credenciales de administrador están codificadas directamente en login.php (en un entorno de producción se recomienda usar un sistema más seguro).
    \item Todas las operaciones de administración verifican la existencia de una sesión válida.
    \item Se implementa validación de datos tanto en el cliente como en el servidor.
    \item Los archivos se validan por tipo y tamaño antes de ser procesados.
\end{itemize}

\subsection{Extensión y Personalización}
Para extender o personalizar el sistema de administración, se pueden considerar las siguientes mejoras:

\begin{itemize}
    \item Implementar un sistema de usuarios con diferentes niveles de acceso.
    \item Agregar funcionalidad para editar canciones existentes.
    \item Implementar un sistema de categorías o etiquetas para las canciones.
    \item Mejorar la seguridad mediante el uso de hashing de contraseñas y tokens CSRF.
    \item Agregar estadísticas de uso y reproducciones de las canciones.
\end{itemize}
